\section{Exercise 6}
\begin{equation*}
UD \approx \frac{log_2|K|}{R_L\cdot log_2|P|}
\end{equation*}
For a substitution cipher where the plaintext is english we have:\\
\begin{tabular}{ll}
$|K|$ &= 26!\\
$|P|$ &= 26\\
$R_L$ &= 0.7
\end{tabular}\\
Substituting this we get:
\begin{equation*}
\begin{split}
UD &\approx \frac{log_2(26!)}{0.7\cdot log_2(26)}\\
UD &\approx \frac{88.4}{0.7\cdot 4.7}\\
UD &\approx \frac{88.4}{3.29}\\
UD &\approx 27\\
\end{split}
\end{equation*}
This means that for a general substitution cipher in normal english, we need
atleast 27 characters to have any chance at sucess.

\section{Exercise 8}
\begin{equation*}
\chi^2=\frac{(o_1-e_1)^2}{e_1} + \frac{(o_2-e_2)^2}{e_2} +
\ldots + \frac{(o_k-e_k)^2}{e_k}
\end{equation*}
\begin{tabular}{ll}
We tested &	1000 bits.\\
We got:   &	542 zerroes\\
		  &	458 ones.\\
We expect &	50/50
\end{tabular}\\
Hypothesis $H_1$ : the numbers are divided evenly with a 5\% significanse.\\
Alternate: reject $H_1$ if $\chi^2 > 3.841$ (1 deg of freedom)\\
\begin{equation*}
\begin{split}
\chi^2	&=\frac{(542-500)^2}{500} + \frac{(458-500)^2}{500}\\
\chi^2	&= 7.052
\end{split}
\end{equation*}
As my $\chi^2$ test exceeds the value I have from my$\chi^2$ table\cite{Prob} I
discard the hypothesis, they are not evenly divided.
\section{Exercise 9}
\begin{tabular}{ll}
We tested &1000 times 2bit-pairs.\\
We got:   &231 times 00\\
		  &271 times 01\\
		  &270 times 10\\
		  &228 times 11.\\
We expect &	25/25/25/25
\end{tabular}\\
Hypothesis $H_1$ : the numbers are divided evenly with a 5\% significanse.\\
Alternate: reject $H_1$ if $\chi^2 > 7.815$ (3 deg of freedom)\\
\begin{equation*}
\begin{split}
\chi^2	&=\frac{(231-250)^2}{250} + \frac{(271-250)^2}{250} +
\frac{(270-250)^2}{250} + \frac{(228-250)^2}{250}\\
\chi^2	&= 6.744
\end{split}
\end{equation*}
As my $\chi^2$ test does not exceed the value I have from my$\chi^2$
table\cite{Prob} I can say that the values are evenly distributed with a 5\%
margin.

\section{Exercise 11}
given the key:
\begin{center}
	\begin{tabular}{ l|cccccccc}
	  p & 0 & 1 & 2 & 3 & 4 & 5 & 6 & 7\\ \hline
	  c & 13 & 4 & 3 & 12 & 1 & 0 & 8 & 10\\
	    ~\\
	  p & 8 & 9 & 10 & 11 & 12 & 13 & 14 & 15\\ \hline
	  c & 14 & 6 & 9 & 15 & 11 & 2 & 5 & 7\\
	\end{tabular}
\end{center}
To encode the 16-number sequence $(3,3,3\dots,3)$ we do:
\begin{lstlisting}
    public static void main(String[] args)
	{
		int[] key = {13,4,3,12,1,0,8,10,14,6,9,15,11,2,5,7};
		String[] key2 ={"1101", "0100", "0011", "1100", "0001", "0000", "1000", "1010", "1110", "0110", "1001", "1111", "1011", "0010", "0101", "0111"};
		String plaintext = "0011";
		
		String init = "0000";
		String plain = "0011";
		for (int i = 0; i < 16; i++)
		{
			plain = XOR(plain, init);
			System.out.print(plain + " ");
			plain = key2[Integer.parseInt(plain, 2)];
			System.out.println(plain);
		}
	}
\end{lstlisting}
and get:\\
\begin{tabular}{cc}
XOR & encrypt\\
0011 &1100\\
1100 &1011\\
1011 &1111\\
1111 &0111\\
0111 &1010\\
1010 &1001\\
1001 &0110\\
0110 &1000\\
1000 &1110\\
1110 &0101\\
0101 &0000\\
0000 &1101\\
1101 &0010\\
0010 &0011\\
0011& 1100\\
1100 &1011
\end{tabular}
