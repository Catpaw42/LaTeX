\documentclass[11pt, a4paper]{article}

%Begin preamble
\usepackage[utf8]{inputenc}
\usepackage{datetime}
\usepackage{graphicx}
\usepackage{fancyhdr}
\usepackage[hidelinks]{hyperref}
\usepackage{lastpage}
\pagestyle{fancyplain}
\fancyhf{}
\lhead{ \fancyplain{}{masl13ab} }
\rhead{ \fancyplain{}{Magnus Brandt Sløgedal} }
\chead{ \fancyplain{}{HjemmeOpg 1} }
\cfoot{ \fancyplain{}{\thepage/\pageref{LastPage}} }
\bibliographystyle{apalike}
\title{Hjemmeopgave 1}
\author{Magnus Brandt Sløgedal\\ masl13ab}
\ddmmyyyydate
\date{\today}
\renewcommand\refname{Referencer}

%End preamble
\begin{document}
\maketitle
\section{Opgave 1}
\subsection*{A1: Hvem hæfter for Kim Jensens krav på kompensation, og hvorfor?}
\textbf{Har Kim Jensen et krav på erstatning mod Dan Klein?}\\
Kim Jensen har her lidt et økonomisk tab, da han i fem uger ikke har kunnet
passe sit arbejde. Der er også både kausalitet og adækvans, da det er
samsynligt og forudseeligt at filk kan falde på et glat gulv, og at det glatte
gulv var medvirkende til at Kim Jensen faldt.\\
Med hensyn til ansvarsgrundlaget kan der argumenteres for simpel uagtsomhed
eller hændelighed.\\
Argument kan laves for at Dan Klein har handlet med simpel uagtsomhed, da han
unlod at tørre olien op, idet han vidste at der var en risiko for at kunder
kunne finde på at bevæge sig ud på lageret, Dette er det ansvarsgrundlag jeg vil
arbejde videre med i opgaven.\\
Et modstående argument, for at der er tale om en hændelig skade, ville være at
olien var spildt ude på lageret hvor Dan Klein er den eneste der burde opholde
sig, og det derfor var forsvarligt for ham at prioritere butikkens drift.\\
Herudfra har Kim Jensen umidelbart et erstatningskrav mod Dan Klein, dog findes
der et særligt ansvarsfrihedsgrundlag da Lars Andersen har tegnet en forsikring
der dækker hans ansatte, Da Dan Klein kun har handlet med simpel uagtsomhed kan
der ikke rettese et krav mod han jf. EAL §19 stk. 3.\\
\\
\textbf{Kan kravet rettes mod arbejdsgiveren Lars Andersen?}\\
For at der skal være et arbejdsgiveransvar skal der gælde:
\begin{itemize}
  \item Er der et ansættelsesforhold?
  \item Har den ansatte handlet culpøst?
  \item Står handlingen i naturlig forbindelse med arbejded?
\end{itemize}
Der er her tale om at Dan Klein er ansat hos Lars Andersen, at Dan Andersen
har handlet culpøst i en unladelse der er en naturlig del af hans arbejde, da
oprydning af ødelagte vare er en del af butiksarbejdet.

Herudfra kan vi konkludere at Lars Andersen hæfter som arbejdsgiver for skaderne
som er overgået Kim Jensen. 

\subsection*{A2: Hvilke krav kan Kim Jensen få dækket, og i hvilket omfang?}
Kim jensen kan få erstatning for de 5 ugers tabt arbejde, og kan også søge
erstatning for svie og smerte jf. EAL §3.

\subsection*{B1: Hvem hæfter for Søren Sørensens erstatningskrav, og hvorfor?}
Søren Sørensen skal også rette sit krav mod Lars Andersen, der også her hæfter
som arbejdsgiver, da der er kausalitet og adækvans i at tøj kan blive ødelagt
når man falder på glasskår. 

\subsection*{C: Hvorvidt kan Lars Andersen kræve betaling for de ituslåede øl
hos Kim Jensen? hos Søren Sørensen? og hos Dan Klein?}
Da Kim Jensen har en særlig ansvarsfrihedsgrund i kraft af Dan Kleins uagtsomhed
kan der ikke rettes et krav mod ham, selv om de t er han der har ødelagt
øl'ne.\\
Søren Sørensen hæfter ikke som arbejdsgiver da Kim ikke har handlet culpøst, og
desuden ikke handlede i naturlig forbindelse med arbejded.\\
Der kan rettes et krav mod Dan Klein, da der er kausalitet og adækvans for at
vare kan gå istykker når kunder falder, med da Dan Klein er dækket af Lars
Andersens forsikring, og Lars Andersen hæfter som arbejdsgiver, bliver Dan Klein
fritaget og Lars Andersen kan kun rette sit krav mod sigselv.

\section{Opgave 2}
\subsection*{A: vedrørende aftalen om leksikonet.}
Det er i denne opgave meget ligetil at se at der er indgået en aftale, så det
første spørgsmål er om dette er en forbrugeraftale eller ej.\\
Vestre Landsret dømte i en lignende sag \footnote{U 1999.285 V.} at: Også en
”lønmodtager der, for egen regning køber arbejdsredskaber til anvendelse i sit
arbejde, anses for en forbruger.”\cite[p.~141]{erhvervsjura}
Ud fra ovenstående anser jeg Lise Lotte Madsen for en forbruger, og der er
derfor tale om en forbrugeraftale.

Dette betyder at forbrugeraftalelovens §6 stk 1, der forbyder dørsalg, samt §7
der siger at alle løfter afgivet af en forbruger under en erhvervsdrivendes
henvendelse i strid med §6 er ugyldige, stiller Lise Lotte Madsen frit iforhold
til hendes aftale med ”Rohan´s Digest”.\\
Skulle der mod forventning ikke være tale om en forbrugeraftale, kan Lise Lotte
Madsen i stedet benytte Aftalelovens §30 stk. 1 og stk. 2, der gør
viljesærklæringer ugyldige hvis de er fremkaldt ved svig, da sælger i dette
tilfælde har givet forkerte oplysninger mht. relativ pris og
markedsundersøgelser, som man må forvente at han ved bedre om.

\subsection*{B: vedrørende abonnementet på tidsskriftet ”Economy”}
Det er min mening at den samme situation som ved leksikonet foreligger her,
dørsalg i strid med Forbrugeraftaleloven, men skulle ”Rohan´s Digest” mod
forventning få ret i deres argumentation for at Lise Lotte Madsen, i aftalen om
tidsskriftet, handlede som erhvervsdrivende, hænger hun på sit abonement da
aftalen er indgået mundtligt og således træder i kraft straks jf. §3 stk. 2.

\bibliography{bib}
\end{document}
