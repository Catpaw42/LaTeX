\documentclass[11pt, a4paper]{article}

%Begin preamble
\usepackage[utf8]{inputenc}
\usepackage{datetime}
\usepackage{graphicx}
\title{Hjemmeopgave 1}
\author{Magnus Brandt Sløgedal}
\ddmmyyyydate
\date{\today}

%End preamble
\begin{document}
\maketitle
\section{Opgave 1}
\subsection*{A}
\textbf{Har Kim Jensen et krav på erstatning mod Dan Klein?}
\begin{itemize}
  \item Tab\\
  Da Kim Jensen har været ude af stand til at passe sit job i 5 uger er der sket
  et økonomisk tab
  \item Ansvarsgrundlag\\
  Et argument kan laves for at Dan Klein har handlet med simpel uagtsomhed, da
  han unlod at tørre olien op, idet han vidste at der var en risiko for at
  kunder kunne finde på at bevæge sig ud på lageret.\\ Det modsatte argument,
  for en hændelig skade, ville være at olien var spildt ude på lageret hvor Dan
  Klein er den eneste der burde opholde sig, og det derfor var forsvarligt for
  ham at prioritere butikkens drift.
  \item Kausalitet\\
  Der formodes at være kausalitet, da det er sansynligt at der er en sammenhæng
  mellem den spildte olie, og Kim Jensens fald. 
  \item Adækvans\\
  Der formodes også at være adækvans da det er påregneligt at folk kan falde på
  et glat gulv.
  \item Særlige ansvarsfrihedgrunde\\
  der er ingen særlige ansvarsfrihedsgrund
\end{itemize}

\textbf{Kan kravet rettes mod arbejdsgiveren?}
\begin{itemize}
  \item Er der et ansættelsesforhold?\\
  Ja der er et ansættelsesforhold mellem Dan Klein og Lars Andersen.
  \item Har den ansatte handlet culpøst?\\
  Ja Dan Andersen har ved sin unladelse handlet culpøst.
  \item Står handlingen i naturlig forbindelse med arbejded?\\
  Ja oprydning af ødelagte vare er en naturlig del af butiksarbejdet.
\end{itemize}

\textbf{Findes der nogen grunde til at fritage den ene eller den anden?}
\begin{itemize}
  \item Da Lars Andersen har tegnet en forsikring, og Dan Klein kun har handlet med
		simpel uagtsomhed, skal kravet jf. EAL §19 stk. 3 rettes mod Lars Andersen.
\end{itemize}

\textbf{Hvilke krav kan Kim Jensen få dækket, og i hvilket omfang?}
5 ugers arbejds tab\\
personskade (svie og smerte)

\subsection*{B}
\textbf{Har Søren Sørensens et krav på erstatning mod Dan Klein?}

\textbf{Har Søren Sørensens et krav på erstatning mod Kim Jensen?}

\textbf{Hvilke krav kan Søren Sørensens få dækket, og i hvilket omfang?}

\subsection*{C}
\textbf{Hvorvidt kan Lars Andersen kræve betaling for de ituslåede øl hos Kim Jensen?}

\textbf{Hvorvidt kan Lars Andersen kræve betaling for de ituslåede øl hos Søren
Sørensen?}

\textbf{Hvorvidt kan Lars Andersen kræve betaling for de ituslåede øl hos Dan Klein?}


\section{Opgave 2}
er der ingået en aftale om leksikonet?

er der ingået en aftale on abonnementet?

Bogen side 141 \begin{quote}
også en lønmodtager der, for egen regning køber
arbejdsredskaber til anvendelse i sit arbejde, anses for en forbruger Jf.
herved U 1999.285 V.
\end{quote}


Forbrugeraftaleloven §6 stk 1 forbyder dørsalg, og §7 diktere at et løfte
afgivet af en forbruger under sådanne omstændigheder ikke er gyldigt
check
datoer, aftale ingået? diskuter om hun er forbruger eller pro


\end{document}
