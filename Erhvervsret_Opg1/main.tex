\documentclass[11pt, a4paper]{article}

%Begin preamble
\usepackage[utf8]{inputenc}
\usepackage{datetime}
\usepackage{graphicx}
\title{Hjemmeopgave 1}
\author{Magnus Brandt Sløgedal}
\ddmmyyyydate
\date{\today}

%End preamble
\begin{document}
\maketitle
\section{Opgave 1}
\subsection*{A1: Hvem hæfter for Kim Jensens krav på kompensation, og hvorfor?}
\textbf{Har Kim Jensen et krav på erstatning mod Dan Klein?}\\
Kim Jensen har her lidt et økonomisk tab, da han i fem uger ikke har kunnet
passe sit arbejde. Der er også både kausalitet og adækvans, da det er
samsynligt og forudseeligt at filk kan falde på et glat gulv, og at det glatte
gulv var medvirkende til at Kim Jensen faldt.\\
Med hensyn til ansvarsgrundlaget kan der argumenteres for simpel uagtsomhed
eller hændelighed.\\
Argument kan laves for at Dan Klein har handlet med simpel uagtsomhed, da han
unlod at tørre olien op, idet han vidste at der var en risiko for at kunder
kunne finde på at bevæge sig ud på lageret, Dette er det ansvarsgrundlag jeg vil
arbejde videre med i opgaven.\\
Et modstående argument, for at der er tale om en hændelig skade, ville være at
olien var spildt ude på lageret hvor Dan Klein er den eneste der burde opholde
sig, og det derfor var forsvarligt for ham at prioritere butikkens drift.\\
Herudfra har Kim Jensen umidelbart et erstatningskrav mod Dan Klein, dog findes
der et særligt ansvarsfrihedsgrundlag da Lars Andersen har tegnet en forsikring
der dækker hans ansatte, Da Dan Klein kun har handlet med simpel uagtsomhed kan
der ikke rettese et krav mod han jf. EAL §19 stk. 3.\\
\\
\textbf{Kan kravet rettes mod arbejdsgiveren Lars Andersen?}\\
For at der skal være et arbejdsgiveransvar skal der gælde:
\begin{itemize}
  \item Er der et ansættelsesforhold?
  \item Har den ansatte handlet culpøst?
  \item Står handlingen i naturlig forbindelse med arbejded?
\end{itemize}
Der er her tale om at Dan Klein er ansat hos Lars Andersen, at Dan Andersen
har handlet culpøst i en unladelse der er en naturlig del af hans arbejde, da
oprydning af ødelagte vare er en del af butiksarbejdet.

Herudfra kan vi konkludere at Lars Andersen hæfter som arbejdsgiver for skaderne
som er overgået Kim Jensen. 

\subsection*{A2: Hvilke krav kan Kim Jensen få dækket, og i hvilket omfang?}
Kim jensen kan få erstatnign for de 5 ugers tabt arbejde, og kan også søge
erstatning for svie og smerte jf. EAL §3.

\subsection*{B1: Hvem hæfter for Søren Sørensens erstatningskrav, og hvorfor?}

\subsection*{B2: I hvilket omfang kan kravet dækkes?}

\subsection*{C: Hvorvidt kan Lars Andersen kræve betaling for de ituslåede øl
hos Kim Jensen? hos Søren Sørensen? og hos Dan Klein?}

\section{Opgave 2}
\subsection*{A: vedrørende abonnementet på tidsskriftet ”Economy”}

\subsection*{B: vedrørende aftalen om leksikonet.}


Bogen side 141 \begin{quote}
Også en lønmodtager der, for egen regning køber
arbejdsredskaber til anvendelse i sit arbejde, anses for en forbruger Jf.
herved U 1999.285 V.
\end{quote}


Forbrugeraftaleloven §6 stk 1 forbyder dørsalg, og §7 diktere at et løfte
afgivet af en forbruger under sådanne omstændigheder ikke er gyldigt
check
datoer, aftale ingået? diskuter om hun er forbruger eller pro


\end{document}
