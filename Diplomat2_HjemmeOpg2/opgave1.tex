For alle reelle tal $a$ er givet de fire vektorer:
\begin{equation*}
v1 = \left(
\begin{array}{c}
1\\
3\\
10 - a
\end{array}
\right)
v2 = \left(
\begin{array}{c}
-1\\
2a - 1\\
-11
\end{array}
\right)
v3 = \left(
\begin{array}{c}
0\\
2a + 1\\
-a-1
\end{array}
\right)
v4 = \left(
\begin{array}{c}
1\\
2-3a\\
a+12
\end{array}
\right)
\end{equation*}
Jeg indsætter først det sidste tal i mit studienummer$(5)$ istedet for $a$.
\begin{equation*}
v1 = \left(
\begin{array}{c}
1\\
3\\
5
\end{array}
\right)
v2 = \left(
\begin{array}{c}
-1\\
9\\
-11
\end{array}
\right)
v3 = \left(
\begin{array}{c}
0\\
11\\
-6
\end{array}
\right)
v4 = \left(
\begin{array}{c}
1\\
-13\\
17
\end{array}
\right)
\end{equation*}
For at vise at vektorerne $v1$, $v2$ og $v3$ er lineært uafhængige skriver jeg
dem op i en matrix:
\begin{equation*}
\begin{bmatrix}
1 & -1 & 0\\
3 & 9 & 11\\
5 & -11 & -6
\end{bmatrix}
\end{equation*}
Og reducere den til echelon-form:\\
$
	R2 = R2 + R1 \cdot -3\\
	R3 = R3 + R1 \cdot -5\\
$
\begin{equation*}
\begin{bmatrix}
1 & -1 & 0\\
0 & 12 & 11\\
0 & -6 & -6
\end{bmatrix}
\end{equation*}
$
	R2 = R2 + R3 \cdot 2\\
	$
	Swap row 2 and 3\\
	$
	R2 = R2 \cdot -\frac{1}{6}\\
$
\begin{equation*}
\begin{bmatrix}
1 & -1 & 0\\
0 & 1 & 1\\
0 & 0 & -1
\end{bmatrix}
\end{equation*}
$
	R3 = R3 \cdot -1\\
	R2 = R2 + R3 \cdot -1\\
	R1 = R1 + R2 \cdot 1\\
$
\begin{equation*}
\begin{bmatrix}
1 & 0 & 0\\
0 & 1 & 0\\
0 & 0 & 1
\end{bmatrix}
\end{equation*}
Vi kan nu se at da matrixen har fuld rang, er vectorerne $v1$, $v2$ og $v3$
lineært uafhængige.

For at opstille $v4$ som linearkombination af de andre tre vektore opstiller jeg
dem i en matrix med $v1..v3$ som koeficienmatix, og $v4$ som resultatvektor:
\begin{equation*}
\begin{bmatrix}
1 & -1 & 0 & 1\\
3 & 9 & 11 & -13\\
5 & -11 & -6 & 17
\end{bmatrix}
\end{equation*}
Herefter reducere jeg med Maples ReduceRowEchelon kommando:
\begin{equation*}
\begin{bmatrix}
1 & 0 & 0 & 7\\
0 & 1 & 0 & 6\\
0 & 0 & 1 & -8
\end{bmatrix}
\end{equation*}
og får resultatet:
\begin{equation*}
v1 = \left(
\begin{array}{c}
7\\
6\\
-8
\end{array}
\right)
\end{equation*}
Hvis jeg ganger dette med de respektive ligninger $v1..v3$ får jeg:
\begin{equation*}
v1 = \left(
\begin{array}{c}
7\\
21\\
35
\end{array}
\right)
v2 = \left(
\begin{array}{c}
-6\\
54\\
-66
\end{array}
\right)
v3 = \left(
\begin{array}{c}
0\\
-88\\
48
\end{array}
\right)
\end{equation*}
Til sidst trækker jeg sammen på tværs (iforhold til de tre ligninger rækkerne
representere) og får:
\begin{equation*}
v4 = \left(
\begin{array}{c}
1\\
-13\\
17
\end{array}
\right)
\end{equation*}
Som var den ønskede $v4$.












