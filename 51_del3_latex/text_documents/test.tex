\section*{BrugerTest}
\addcontentsline{toc}{subsection}{BrugerTest}
Vi har kørt nogle bruger tests, for at se om spillet opfører sig som forventet.
Vi har forsøgt at finde de fejl som gør at spillet af en eller anden grund giver
et forkert output, så som at spiller ikke går fallit når han skal, men får lov
at spille videre. Under \textbf{Bilag} kan man se screenshots fra en
brugertest.\\
\indent Under gennemførelsen af en brugertest, så det ud som at en vinder ikke
ikke blev præsenteret for spillerne. For at checke om controlleren kørte loopet
som finder vinderen, addede vi nogle linjer som blev printet hvis dele af loopen
blev kørt. Da vi så kørte den sidste brugertest fandt vi ud af at vinderen altid
spiller 1 selv om denne spiller var gået fallit
\textbf{\hyperref[bilag9]{Bilag9}}.
Fejlen var at da vi havde tilføjet den linje som blev printet, så havde vi glemt at omslutte if
statement med “\{“ således at alle spillere for den sags skyld kunne være
vindere. Ellers har vi kørt en fuldkommen brugertest hvor vi kommer igennem alle
scenarier vi kunne komme i tanker om. Bilag 1-9 er de scenarier vi forventer når
vi lander på et felt. Hvis man lander på et felt som ikke har en ejer, kan man
vælge at købe det, eller lade være \textbf{\hyperref[bilag1]{Bilag1}}.\\
\indent Hvis en spiller lander på et felt som er ejet af en anden, så får man
ikke mulighed for at købe det, men skal betale til ejeren
\textbf{\hyperref[bilag2]{Bilag2}}. Man kan også se at navnet på feltet og
navnet som bliver præsenteret som ejer af feltet er det samme. Det er den gule bil
(Christian) som er endt på Magnus felt og skal betale ham leje.\\
\indent Når man lander på det ene af tax felterne skal man have 2 muligheder for
at betale (enten 10\% eller 4000kr) \textbf{\hyperref[bilag7]{Bilag7}}. Mens det
andet tax felt kun skal kræve et bestemt beløb
\textbf{\hyperref[bilag3]{Bilag3}}. Disse felter får man ikke mulighed for at
købe, ligesom refugee heller ikke kan købes, men her får man penge
\textbf{\hyperref[bilag4]{Bilag4}}.\\
\indent Et andet scenarie er hvis du lander på dit eget felt, så skal der ikke
ske spilleren noget. Spilleren skal kun få at vide at han er landet på sit eget
felt \textbf{\hyperref[bilag5]{Bilag5}}, her kan nævnes at vi har gjort således
at felterne på GUI’en opdateres så man kan se hvem er ejer af feltet. Det er også
aktuelt når en spiller går fallit, så bliver den spiller fjernet som ejer af de
felter han havde og andre spillere får mulighed for at købe det
\textbf{\hyperref[bilag6]{Bilag6}}.\\
\indent Vi skulle også håndtere at en spiller ikke kan købe et felt for så at gå
fallit fordi han ikke har penge nok, og dette håndtere spillet
\textbf{\hyperref[bilag8]{Bilag8}}.
Spilleren får at vide at han ikke har penge nok, og spillet kører videre. dog
kan spilleren godt gå fallit når han lander på tax feltet hvor man vælge mellem
10\% og 4000 kr. Hvis han har råd til det ene, men ikke det andet, kan han
alligevel gå fallit ved forkert valg.\\
\indent To fejl som vi fandt ved brugertest, men som ikke er rettet, er hvis man 
skriver et grotesk langt navn for en spiller. Navnet bliver så langt at når det er
spillerens tur, dækker navnet “OK” knappen, og spillet kan ikke fortsættes
\textbf{\hyperref[bilag10]{Bilag10}}. Det skal altså være et virkelig, virkelig
langt navn. Hvis spillerne beslutter sig for at have samme navn, så bliver bliver to spillere
oprettet med hver sin konto, men på GUI’en har de samme farve bil
\textbf{\hyperref[bilag11]{Bilag11}}, altså er der kun en bil på brættet, men
som alligevel har to destinationer. Man kan derfor kun se bilen og balancen for den
ene spiller ad gangen.
\section*{Black Box (Junit)}
\addcontentsline{toc}{subsection}{Black Box (Junit)}
Vi har lavet en JUnit test class FieldJUnitTest, der tester LandOnField for alle
vores forskellige Field-klasser. Den opretter først sin egen test GUI, den skal
bruges da der er brugerinput i LandOnField, og derefter kører den en @test for
hver Field-type. Indlejret i disse test er også nogle test på ting fra Ownable,
såsom at gå fallit og lande på ens eget felt.\\
\indent Vi har kørt FieldJUnitTest løbene og brugt den til at rette adskillige
logiske fejl, til et punkt hvor FieldJUnitTest nu kører fejlfrit.
\section*{FURPS+}
\addcontentsline{toc}{subsection}{FURPS+}
FURPS+ er en forkortelse for Functionality, Usability, Reliability, Performance,
Supportability. FURPS+ er en god checkliste for at opnå et så godt program som
muligt, og ikke glemme nogle vigtige dele.
\begin{itemize}
  \item \textbf{functionality:} Vi har lavet et funktionelt spil, som opfylder
  alle kode mæssige krav, som for eksempel at bruge arv når vi programmerede felt
  klasserne og metoden landOnField i klassen Field.
  \item \textbf{Usability:} Spillet er brugervenligt, og der behøves ikke at
  læses nogen brugervejledning for at kunne gennemføre spillet. Dog skal man
  kende noget til matador regler for at forstå spillet, men du gennem spillet
  skal du kun trykke “OK” eller får du to valgmuligheder, hvor der er beskrevet
  hvad de betyder.
  \item \textbf{Reliability:} Gennem bruger tests og en stor J-Unit test, har vi
  elimineret alle de fejl, som opstår gennem de forventede scenarier. Som et af
  kravene har vi med J-Unit testet feltet af typen fleet for:
  \begin{itemize}
    \item at spiller ikke får fratrukket beløb når han lander på felt som han
    selv ejer.
    \item Når en spiller skal betale til en anden spiller, at den rigtige
    spiller får bestemt beløb og den anden fratrukket samme beløb.
    \item Når en spiller køber en fleet, bliver prisen af fleet fratrukket
    spillerens konto, og spiller bliver sat som ejer af feltet.
    \item Hvis en spiller er ejer af to fleets, så får den spiller som lander på
    en af hans fleet, fratrukket et beløb som svarer til $250 \cdot 2^(antal af
    fleets)$. Og ejeren tilføjet samme beløb.
    \item I tilfælde af at en spiller ikke har nok kapital ved landing på en
    andens spillers fleet, så bliver spillerens $balance = 0$, og
    $active = false$ (spilleren bliver inaktiv og er ikke mere del af
    spillet). Derefter bliver de penge som spilleren havde, overført til ejeren
    af fleet.
  \end{itemize}
  Så vi har et pålideligt spil, som ikke lukker ned uforventet.
  \item \textbf{Performance:} Dette har ikke været relevant faktor i vores spil.
  Spillet bruger så få ressourcer at vi har ikke haft brug for kørtids-analyser
  og optimering, spillet optager heller ikke mærkbar disk-space.
  \item \textbf{Supportability:} Vi har gjort en hel del ud af at fremtidssikre
  koden, så vi kan genbruge så meget som muligt af koden senere. For eksempel
  har vi undgået “hard coding”, og al tekst i spillet bliver gennem identifiers
  oversat, ved hjælp af decorator og properties filer, til meningsfyldt tekst.
  Properties filerne indeholder al tekst, og det gør det nemt at have overblik
  og ændre på teksten. Vi stødte dog på et problem, som vi løste med gøre
  felterne til subcontrollere, det gør koden lidt sværere at vedligeholde, da
  små ændringer i Decorator kan gøre at vi må gennemgå gamecontroller samt alle
  subcontrollerne for fejl.
  \item \textbf{+ (Plusset):} Spillet kræver at computeren har og kan køre en
  opdateret version af java og at mus og tastatur er tilkoblet. Spillet har
  ingen lyd, så det er ikke en nødvendighed. Ellers er der ikke nogen
  nævneværdige krav som spillet har til computeren.
\end{itemize}