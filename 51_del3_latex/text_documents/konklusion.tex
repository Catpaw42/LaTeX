Vi har alt i alt fremstillet et spil, der funktionelt set lever op til
specifikationerne - og tilføjer ekstra funktionalitet. Det kan spilles med
begrænsede forkundskaber, om end et kendskab til reglerne er nødvendigt for at
få det optimale udbytte. Det er pålideligt, men mangler et tjek for
brugernavnenes længde - så det er muligt at ødelægge brugeroplevelsen for sig
selv ved at indtaste grotesk lange brugernavne.\\
\indent Vi har taget nogle designbeslutninger, der skaber højere kobling og
bryder med BCE paradigmet. Det giver en overskuelig kode, men skaber det problem at det er
sværere at vedligeholde, idet ændringer i Decorator kan skabe behov for
ændringer i flere forskellige klasser.\\
\indent Idet formålet var at undgå for meget kode i vores GameController, kan
man forsøge at opnå det samme ved at indføre en dedikeret FieldController, der
delegeres ansvaret for sub use casen ‘Land on Field’. Vores dobbelte kobling
mellem Player og felterne, kan evt. løses ved at indføre et
‘tinglysningsregister’ - en (singleton) klasse, der kun har til opgave at holde
styr på ejerskabet af fields.
