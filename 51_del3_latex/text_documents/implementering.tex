Komplekse udsnit af kildekode. Implementering af arv. Kode evt. i bilag
Vores implementering afspejler vores klassediagram.
\section*{Klasser}
\addcontentsline{toc}{subsection}{Klasser}

\subsection*{GameController}
Er en facadecontroller - delegerer så meget ansvar som muligt videre (jvf.
GRASP). Holder således kun de nødvendige instances af koblede klasser. Som nævnt
gør vi en undtagelse og lader Gamecontrolleren holde en int, der angiver hvor
mange spillere, der er tilbage i spillet. Dette er gjort for at undgå ‘tung’
kode, hvor man er nødt til at iterere over spillerne for at finde ud af hvor
mange der er tilbage.
\subsection*{Decorator}
Decorator er designet efter decorator-patternet, dens primære funktion er at
dekorere output til GUI’en. Ligesom GUI’en er Decorator en static klasse, så vi
kan tilgå den alle steder fra, dette er et bevidst designvalg da vi har brudt
med Border - Controller - Entity på dette punkt og udskriver text fra vores
Field klasser.\\
\indent Decorator indeholder jævnføre design-patternet kopier af GUI’s metoder,
disse er designet via. varargs, så tekststrenge kan deles op og oversættes
individuelt før de bliver sat sammen og passet til de relevante GUI metoder. Til
at oversætte tekst bruger vi en java.util.Properties klasse, med tilhørende
.properties filer for hvert sprog i spillet, der er blevet extended for at give
os en specifik handling af den særlige case hvor properties returnere null (den
ønskede key findes ikke).
\subsection*{Board}
Board er en simpel klasse der indeholder vores fields i et array, dens
constructor tager en DiceCup reference, som den passer videre til LaborCamp, og
opretter derefter alle vores fields. Til sidst bruger den hjælper klassen
Shuffler, til at blande felterne, så boardet bliver nyt for hvert spil.
\subsection*{Field}
Field er roden (super class) for vores felt-klasser. Den er en abstrakt klasse
og har nogle abstrakte metoder.\\
\indent Abstrakt betyder for klasser at de ikke kan instantieres, og det betyder
for metoder at de ikke bliver implementeret i denne klasse (ingen metode body).
Til gengæld bliver alle subklasser tvunget til at implementere samtlige
abstrakte metoder.\\
\indent Field har også en metoden decoratorMessage, denne metode er ligeledes
implementeret i samtlige subklasser, men er ikke abstrakt da vi skal bruge den
implementation i Field.\\
\indent decoratorMessage er designet til at bruge nedarvningen, (gennem kald til
super.decoratorMessage) til at bestemme hvilken tekst der skal sendes til
decoratoren. Den er lidt specielt opbygget da den, i stedet for at konkatenere
strengene og returnere dem som en string, opbygger et String array som
returneres til Decorator som et vararg.
\subsection*{Ownable}
Ownable er en abstract subklasses af field, dens hovedformål er at implementere
landOnfield for alle klasser der kan ejes. Dette er gjort fordi metoden
landOnfield er den samme for alle felter der kan købes, den eneste forskel er
lejen. Lejen beregnes derfor af en abstrakt getRent metode som sub-klasserne
implementerer.
\subsection*{Fleet}
Fleet et felt og er en subklasse af Ownable, der er karakteriseret ved, at man
opkræver mere i leje pr gang nogen lander på det, jo flere Fleets, man ejer.
Nøjagtigt ligesom vi kender færgerne fra matador. Formlen går $(2^(antal af
fleets)) \cdot 250$. Der ved opkræver man 500 pr gang, hvis man ejer 1 færge;
1000, hvis man ejer 2; 2000, hvis man ejer 3; og 4000, hvis man ejer alle fire
færger.\\
\indent Vores feltklasser holder som sagt en smule kontrol, da metoden
landOnField() har en use-relation til GUI’en.\\
\indent Constructoren til Fleet holder en String title, int price og int
baseRent.
\subsection*{LaborCamp}
LaborCamp er et felt og er en subklasse af Ownable, der er karakteriseret ved,
at man betaler 100 gange øjnene af den uheldiges kast. Desuden vil man betale
dobbelt, hvis en spiller ejer begge LaborCamps. Borset fra prisen, er det
ækvivalenten til sodavandsfabrikkerne i Matador. Formlen for den opkrævede leje
er ((antal LaborCamps ejet af opkræveren) * (kastet) * 100).\\
\indent Constructoren indeholder en String title, int price og int baseRent.
\subsection*{Territory}
Territory er et felt og er en subklasse af Ownable. Det er det mest normale af
de forskellige slags felter, i det der er 11 af dem, og det er meget ligefrem i
forhold til nogle af de andre felter, i det, at man betaler en fast pris, hver
gang man lander på dem. De varierer i pris og leje, men er konstante. Selve
constructeren holder en String title, en int price, og en int rent.
\subsection*{Tax}
Tax er en subklasse af Field, og er et felt, der ikke er en Ownable. Det er et
felt, hvor man betaler skat, hvis man lander der. Skatfelterne fungerer ligesom
i Matador. Der er to af dem, og på det ene bliver der betalt et fast beløb af
2000, og på det andet kan man vælge mellem at betale 4000, eller 10\% af sin
samlede formue. Constructoren holder en String title, int taxAmount og int
taxRate. Her handler det om at have lidt hovedregning, da det forkerte valg
kan koste en dyrt.
\subsection*{Refuge}
Refuge er et felt og en subklasse af klassen Field. Hvor man modtager 500 eller
5000 for at lande på feltet. Constructoren indeholder en private attribute (int
bonus) og en String title.  Derudover har den nedarvet metoden LandOnField(),
som fortæller GUI at spilleren har landet på Refuge og fortæller hvor meget
attributen bonus skal deposit til playerAccount. Derudover har den også en
String$[]$ decoratorMessage og en toString() metode, som returnerer ($title +
“bonus:” + bonus$).
\subsection*{Die}
Klassen Die har to attributter (faceValue: Int og NUM\_SIDES: Int). Den har et
objekt der refererer til klassen Random, som bruges til at generere en
pseudotilfældig integer med variablen NUM\_SIDES og en integer med variablen
faceValue, der holder terningens værdi.\\
\indent Klassen Die er udvalgt fra det forrige terningespil med den mest
interessante metode roll(), der henter en pseudo tilfældig integer mellem 0 og 5
fra objektet Random, hvor faceValue sætter denne random til {++1} og derefter
returnerer faceValue.
\subsection*{DiceCup}
Klassen indeholder metoden rollDice og getterne getSum, getDiceFaceValues og
getTwoOfAKind, hvor der er også er en toString, som dog ikke bliver anvendt i
spillet. DiceCup constructoren instantierer de to terninger som skal bruges i
spillet.\\
\indent Når GameControlleren beder om at få kastet terningerne kører DiceCup
metoden rollDice, så  summen af de to terninger bliver beregnet. Når summen af
de netop kastede terninger skal  bruges, bruger GameController getSum, som
returnere summen. Fordi at GUI’en også skal vise øjnene pa de to terninger,
indeholder DiceCup ogsa getDiceFaceValues som returnere et array af de to
terningers øjne, som GameControlleren så kan tilkalde. Derudover har vi også en
getter til TwoOfAKind (getTwoOfAKind). Denne indeholder ikke selve reglerne til
spillet, men giver GameControlleren muligheden for at kunne tilkalde og få at
vide om terningerne har slået to ens, og hvad der er slået to ens af.
getTwoOfAKind returnerer derfor enten 0, hvis  terningerne ikke er lige ellers
1 for to 1’ere og 2 for to 2’ere osv.\\
\indent String toString kan blive brugt hvis man vil vide terningernes værdier.
Den kan være nyttig hvis man ikke kører spillet igennem en GUI, så bliver en
String returneret med forklaring på hvilke værdier terningerne har.
\subsection*{Player}
Player har til formål at opbevare data for den enkelte spiller. Den indeholder
variable for navn og placering på brættet, en account til at styre spillerens
balance, en liste med alle felter som spilleren ejer og en bools variabel der
styrer om man stadig er med i spillet. Herudover har player alle nødvendige
get/set metoder og en toString.\\
\indent Player har også en goBroke metode, der er bruges til at rydde op efter
en spiller der er gået falit. goBroke sætter spillerens formue til 0, fjerner
ham som owner af alle fields, og sætter den boolske “active” variabel til false.
\subsection*{Account}
Klassen Account har atributten int, der modsvarer pengebeholdningen. Den har
derudover en constructor, der opretter en account med et beløb og en default
constructor, som opretter en account med standardbeløbet 30.000.\\
\indent Account var i den tidligere rapport specificeret til at den ikke måtte
få en negativ score. Dette er fortsat, men nu er spilleren bare identificeret
som bankerot. Derfor er der implementeret tjek for negative beløb i metoderne.
Disse tjek er behandlet med exceptions. Derudover har klassen Account en
standard getScore og en setScore, der indeholder et tjek for om metoden forsøger
at kalde med en negativ int.\\
\indent Metoden deposit tjekker for om den modtager en negativ int og for om
score vil gå over maksimumværdien for int. Metoden withdraw kontrollerer for en
negativ int og for om score vil gå under 0. toString er autogenereret af eclipse.
