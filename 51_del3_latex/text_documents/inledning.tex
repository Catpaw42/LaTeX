I IOOuterActive har vi fået en ny ordre fra kunden, som involvere at udvide det
sidste spil vi lavede (CDIO.del2)til et rigtigt brætspil, med nye felt typer,
samt at brug af polymorfi og muligheden for at købe felter.\\
\indent Denne rapport er skrevet til personer med en generel viden omkring UML
(Unified Modeling Language) og programmeringssproget Java. Målet med denne
rapport er at dokumentere vores arbejde på denne ordre i IOOuterActive.\\
\indent Til udarbejdning af rapporten og programmet har vi brugt følgende
værktøjer:
\begin{itemize}
  \item Eclipse - Vores valgte development environment.
  \item GitHub - Vores version control system of choice.
  \item Google Docs - for at synkronisere rapportskrivningen.
  \item LaTeX - For at få et pænere final product.
  \item Software Ideas Modeller - For vores UML-diagrammer.
\end{itemize}
Rapporten beskriver følgende emner:
\begin{itemize}
  \item Krav - Vi har foretaget en kravspecificering og en use-case analyse,
  hvor vi kommer ind på en fully dressed use-case beskrivelse  af  “Land on fleet”,
  for at give os en bedre forståelse af kundens ønsker, og omfanget af
  opgaven.
  \item Design - Vi har dokumenteret vores design i UML form for at give os et
  overblik over programmet og fastlægge placeringen af koden. vi bestræber os på
  at overholde GRASP koncepterne i designet.
  \item Implementering - Vi har implementeret det valgte design. Dette er blevet
  dokumenteret i rapporten ved hjælp af beskrivelser af alle klasserne, en
  forklaring af konceptet arv i en objekt orienteret sammenhæng samt termet
  abstract.
  \item Test - Vi har testet programmet, både manuelt og via JUnit, og benyttet
  FURPS+ for at prøve at sikre en god maintenance og operation af programmet.
\end{itemize}